% Options for packages loaded elsewhere
\PassOptionsToPackage{unicode}{hyperref}
\PassOptionsToPackage{hyphens}{url}
%
\documentclass[]{article}
\usepackage{amsmath,amssymb}
\usepackage{lmodern}
\usepackage{iftex}
\ifPDFTeX
  \usepackage[T1]{fontenc}
  \usepackage[utf8]{inputenc}
  \usepackage{textcomp} % provide euro and other symbols
\else % if luatex or xetex
  \usepackage{unicode-math}
  \defaultfontfeatures{Scale=MatchLowercase}
  \defaultfontfeatures[\rmfamily]{Ligatures=TeX,Scale=1}
\fi
% Use upquote if available, for straight quotes in verbatim environments
\IfFileExists{upquote.sty}{\usepackage{upquote}}{}
\IfFileExists{microtype.sty}{% use microtype if available
  \usepackage[]{microtype}
  \UseMicrotypeSet[protrusion]{basicmath} % disable protrusion for tt fonts
}{}
\makeatletter
\@ifundefined{KOMAClassName}{% if non-KOMA class
  \IfFileExists{parskip.sty}{%
    \usepackage{parskip}
  }{% else
    \setlength{\parindent}{0pt}
    \setlength{\parskip}{6pt plus 2pt minus 1pt}}
}{% if KOMA class
  \KOMAoptions{parskip=half}}
\makeatother
\usepackage{xcolor}
\IfFileExists{xurl.sty}{\usepackage{xurl}}{} % add URL line breaks if available
\IfFileExists{bookmark.sty}{\usepackage{bookmark}}{\usepackage{hyperref}}
\hypersetup{
  hidelinks,
  pdfcreator={LaTeX via pandoc}}
\urlstyle{same} % disable monospaced font for URLs
\usepackage{longtable,booktabs,array}
\usepackage{calc} % for calculating minipage widths
% Correct order of tables after \paragraph or \subparagraph
\usepackage{etoolbox}
\makeatletter
\patchcmd\longtable{\par}{\if@noskipsec\mbox{}\fi\par}{}{}
\makeatother
% Allow footnotes in longtable head/foot
\IfFileExists{footnotehyper.sty}{\usepackage{footnotehyper}}{\usepackage{footnote}}
\makesavenoteenv{longtable}
\usepackage[normalem]{ulem}
% Avoid problems with \sout in headers with hyperref
\pdfstringdefDisableCommands{\renewcommand{\sout}{}}
\setlength{\emergencystretch}{3em} % prevent overfull lines
\providecommand{\tightlist}{%
  \setlength{\itemsep}{0pt}\setlength{\parskip}{0pt}}
\setcounter{secnumdepth}{-\maxdimen} % remove section numbering
\ifLuaTeX
  \usepackage{selnolig}  % disable illegal ligatures
\fi

\author{}
\date{}

\begin{document}

\hypertarget{anuxe1lisis-cruxedtico-y-doctrinal-del-ruxe9gimen-de-tutela-laboral-argentina-en-contextos-de-crisis-y-sucesiuxf3n-empresarial-el-caso-red-chamber-argentina-y-la-ex-alpesca-chubut}{%
\section{Análisis Crítico y Doctrinal del Régimen de Tutela Laboral
Argentina en Contextos de Crisis y Sucesión Empresarial: El Caso Red
Chamber Argentina y la Ex-Alpesca
(Chubut)}\label{anuxe1lisis-cruxedtico-y-doctrinal-del-ruxe9gimen-de-tutela-laboral-argentina-en-contextos-de-crisis-y-sucesiuxf3n-empresarial-el-caso-red-chamber-argentina-y-la-ex-alpesca-chubut}}

\hypertarget{i.-introducciuxf3n-y-encuadre-doctrinario-del-conflicto}{%
\subsection{I. Introducción y Encuadre Doctrinario del
Conflicto}\label{i.-introducciuxf3n-y-encuadre-doctrinario-del-conflicto}}

\hypertarget{a.-presentaciuxf3n-del-caso-red-chamber-argentina-rca-y-el-contexto-pataguxf3nico}{%
\subsubsection{A. Presentación del Caso Red Chamber Argentina (RCA) y el
Contexto
Patagónico}\label{a.-presentaciuxf3n-del-caso-red-chamber-argentina-rca-y-el-contexto-pataguxf3nico}}

El conflicto suscitado en torno a Red Chamber Argentina S.A. (RCA),
operadora de los activos de la ex-Alpesca en Puerto Madryn, Provincia
del Chubut, representa un caso de estudio crucial sobre la aplicación
del derecho laboral argentino en contextos de alta volatilidad
industrial y fuerte intervención estatal. El conflicto se origina en
denuncias por incumplimiento de derechos laborales que escalaron hasta
la decisión del Gobierno Provincial de extinguir el contrato de
concesión \textsuperscript{1}, poniendo en juego la continuidad de una
fuente de empleo considerada de utilidad pública.\textsuperscript{2}

Este escenario no se limita a una disputa contractual, sino que es un
conflicto sistémico donde se articulan el régimen de trabajo (Ley
20.744), el derecho colectivo (CCT de Pesca/Alimentación)
\textsuperscript{3}, y la legislación provincial de emergencia (Ley I N°
527).\textsuperscript{2} La relevancia del caso radica en la presión
ejercida por la tutela laboral para garantizar la estabilidad de los
trabajadores a través de sucesivos cambios de operador (de Alpesca a
RCA, y luego a Profand-Consermar).\textsuperscript{1}

\hypertarget{b.-fundamentos-juruxeddico-constitucionales-del-derecho-laboral-argentino}{%
\subsubsection{B. Fundamentos Jurídico-Constitucionales del Derecho
Laboral
Argentino}\label{b.-fundamentos-juruxeddico-constitucionales-del-derecho-laboral-argentino}}

El análisis se enmarca bajo los principios rectores del derecho laboral
argentino, establecidos primariamente por la Ley de Contrato de Trabajo
(LCT) N° 20.744 \textsuperscript{5}, en consonancia con el Artículo 14
bis de la Constitución Nacional.

\hypertarget{el-principio-protectorio-y-la-tutela-reforzada-in-dubio-pro-operario}{%
\paragraph{\texorpdfstring{1. El Principio Protectorio y la Tutela
Reforzada (\emph{\textbf{In Dubio Pro
Operario}})}{1. El Principio Protectorio y la Tutela Reforzada (In Dubio Pro Operario)}}\label{el-principio-protectorio-y-la-tutela-reforzada-in-dubio-pro-operario}}

La LCT parte de la premisa de que el trabajador y el empleador no se
encuentran en un plano de igualdad fáctica \textsuperscript{6}, razón
por la cual la ley busca compensar esa asimetría de poder mediante el

\textbf{Principio Protectorio}. Este principio se manifiesta en la regla
fundamental de la continuidad del contrato, según la cual, en caso de
duda, las situaciones deben resolverse siempre a favor de la
subsistencia de la relación laboral.\textsuperscript{7} Este postulado
fue elevado a una condición contractual en la sucesión de los bienes
expropiados de Alpesca, garantizando la antigüedad y las condiciones de
la plantilla formal en la transferencia de RCA a
Profand.\textsuperscript{1}

\hypertarget{el-orden-puxfablico-laboral-nulidad-de-actos-en-fraude-a-la-ley}{%
\paragraph{2. El Orden Público Laboral: Nulidad de Actos en Fraude a la
Ley}\label{el-orden-puxfablico-laboral-nulidad-de-actos-en-fraude-a-la-ley}}

El régimen laboral establece un piso de derechos mínimos inderogables.
Las cláusulas contractuales que modifiquen en perjuicio del trabajador
normas imperativas consagradas por leyes o convenciones colectivas de
trabajo son consideradas nulas y se entienden sustituidas de pleno
derecho por las normas protectoras.\textsuperscript{8} Esta potestad de
orden público es crucial para combatir el fraude, incluyendo la
prohibición del objeto contractual, la cual está siempre dirigida al
empleador cuando se veda el empleo en determinadas tareas o
condiciones.\textsuperscript{7}

\hypertarget{ii.-marco-regulatorio-triple-lct-cct-y-derecho-administrativo-pesquero}{%
\subsection{II. Marco Regulatorio Triple: LCT, CCT y Derecho
Administrativo
Pesquero}\label{ii.-marco-regulatorio-triple-lct-cct-y-derecho-administrativo-pesquero}}

La regulación de la actividad pesquera industrial en Argentina exige la
superposición de tres estratos normativos: la ley general (LCT), el
convenio colectivo sectorial (CCT) y la ley administrativa de recursos
(Ley Federal de Pesca).

\hypertarget{a.-la-ley-de-contrato-de-trabajo-lct-20.744-como-muxednimo-inderogable}{%
\subsubsection{A. La Ley de Contrato de Trabajo (LCT 20.744) como Mínimo
Inderogable}\label{a.-la-ley-de-contrato-de-trabajo-lct-20.744-como-muxednimo-inderogable}}

La LCT establece obligaciones esenciales del empleador que, de
incumplirse, configuran la base de la denuncia contra RCA.

\hypertarget{el-deber-de-seguridad-higiene-y-protecciuxf3n-integral-art.-75-lct}{%
\paragraph{1. El Deber de Seguridad, Higiene y Protección Integral (Art.
75
LCT)}\label{el-deber-de-seguridad-higiene-y-protecciuxf3n-integral-art.-75-lct}}

El Artículo 75 de la LCT impone al empleador un deber de protección
activo, obligándolo a adoptar las medidas necesarias para proteger la
integridad psicofísica y la dignidad de los trabajadores, garantizando
condiciones de trabajo dignas y seguras.\textsuperscript{9} El
incumplimiento de estas obligaciones puede constituir una injuria grave
que justifique un despido indirecto o, como se observó en el caso RCA,
una causal suficiente para la intervención administrativa y la rescisión
de la concesión por parte del Estado, al validarse las denuncias de
incumplimiento.\textsuperscript{1}

\hypertarget{b.-el-ruxe9gimen-colectivo-particularidades-de-la-pesca-industrial}{%
\subsubsection{B. El Régimen Colectivo: Particularidades de la Pesca
Industrial}\label{b.-el-ruxe9gimen-colectivo-particularidades-de-la-pesca-industrial}}

La actividad de procesamiento de pescado se rige por Convenios
Colectivos de Trabajo específicos.

\hypertarget{aplicaciuxf3n-del-cct-para-plantas-de-procesamiento-ej.-cct-37204}{%
\paragraph{1. Aplicación del CCT para Plantas de Procesamiento (Ej. CCT
372/04)}\label{aplicaciuxf3n-del-cct-para-plantas-de-procesamiento-ej.-cct-37204}}

Los trabajadores de plantas de procesamiento y elaboración de productos
frescos o congelados derivados de la pesca están comprendidos
típicamente bajo el Convenio Colectivo de Trabajo 372/2004 (Alimentación
- Pesca, Procesamiento y Elaboración de Derivados).\textsuperscript{3}
Este convenio, con un ámbito de aplicación que excluye el partido de
General Pueyrredón pero abarca el resto del territorio nacional y un
estimado de 17.000 beneficiarios \textsuperscript{3}, establece las
condiciones de trabajo, categorías y salarios específicos, superando el
piso mínimo de la LCT.

\hypertarget{contexto-cruxedtico-de-los-convenios-colectivos-sectoriales}{%
\paragraph{2. Contexto Crítico de los Convenios Colectivos
Sectoriales}\label{contexto-cruxedtico-de-los-convenios-colectivos-sectoriales}}

Un factor relevante en la dinámica de conflictos sectoriales es la
antigüedad de algunos CCTs. El sector empresario ha señalado la
necesidad de revisar los Convenios Colectivos de Trabajo, algunos con
más de 30 años de vigencia, con el fin de mejorar la competitividad de
la industria.\textsuperscript{11} Esta presión por la modernización
regulatoria puede generar un fuerte incentivo para que las empresas,
percibiendo los CCTs existentes como rígidos o costosos, busquen
mecanismos de evasión, como la subcontratación fraudulenta o la
interposición de terceros, para aplicar regímenes o salarios menos
onerosos, desafiando así la justicia protectora de la ley.

\hypertarget{c.-la-ley-federal-de-pesca-ley-24.922-condicionamiento-laboral-a-las-concesiones}{%
\subsubsection{C. La Ley Federal de Pesca (Ley 24.922): Condicionamiento
Laboral a las
Concesiones}\label{c.-la-ley-federal-de-pesca-ley-24.922-condicionamiento-laboral-a-las-concesiones}}

La actividad pesquera está regulada no solo por normas laborales sino
también por derecho administrativo de fomento y conservación.

\hypertarget{la-vinculaciuxf3n-de-los-permisos-de-largo-plazo-a-la-obligaciuxf3n-de-procesamiento}{%
\paragraph{1. La vinculación de los permisos de largo plazo a la
obligación de
procesamiento}\label{la-vinculaciuxf3n-de-los-permisos-de-largo-plazo-a-la-obligaciuxf3n-de-procesamiento}}

La Ley Federal de Pesca 24.922 fomenta el ejercicio de la pesca marítima
\textsuperscript{12} y establece que los permisos de pesca pueden
otorgarse por plazos de hasta 30 años a buques pertenecientes a empresas
que posean

\textbf{instalaciones de procesamiento radicadas en el territorio
nacional y que elaboren productos pesqueros en forma
continuada}.\textsuperscript{12}

\hypertarget{implicancia-los-derechos-laborales-como-condiciuxf3n-de-licencia}{%
\paragraph{2. Implicancia: Los derechos laborales como condición de
licencia}\label{implicancia-los-derechos-laborales-como-condiciuxf3n-de-licencia}}

El cumplimiento de las obligaciones laborales, el mantenimiento de la
planta operativa y la continuidad de la elaboración no son meros asuntos
internos, sino que se convierten en \textbf{condiciones esenciales} para
mantener el acceso al recurso pesquero. Por lo tanto, el incumplimiento
laboral grave por parte de RCA, validado por la auditoría provincial, no
solo genera multas o demandas, sino que justifica la pérdida del activo
estratégico más valioso: la concesión de explotación.\textsuperscript{1}
De esta manera, el derecho laboral funciona como un mecanismo de control
estatal sobre el cumplimiento de las obligaciones de fomento industrial,
reforzando la aplicación de la ley.

\hypertarget{iii.-la-compleja-sucesiuxf3n-patronal-y-la-aplicaciuxf3n-de-la-responsabilidad-solidaria}{%
\subsection{III. La Compleja Sucesión Patronal y la Aplicación de la
Responsabilidad
Solidaria}\label{iii.-la-compleja-sucesiuxf3n-patronal-y-la-aplicaciuxf3n-de-la-responsabilidad-solidaria}}

El caso de RCA y la ex-Alpesca ilustra la aplicación del principio de
continuidad laboral en una situación de crisis empresarial agravada por
la intervención política sobre bienes expropiados.

\hypertarget{a.-la-transferencia-del-establecimiento-art.-225-lct-en-el-contexto-de-la-expropiaciuxf3n}{%
\subsubsection{A. La Transferencia del Establecimiento (Art. 225 LCT) en
el Contexto de la
Expropiación}\label{a.-la-transferencia-del-establecimiento-art.-225-lct-en-el-contexto-de-la-expropiaciuxf3n}}

El Artículo 225 de la LCT establece que la transferencia o cesión de un
establecimiento mantiene la relación de trabajo con el adquirente,
conservando los trabajadores la antigüedad y todos los derechos
derivados del contrato anterior.

\hypertarget{aplicaciuxf3n-al-caso-rca-y-profand}{%
\paragraph{1. Aplicación al Caso RCA y
Profand}\label{aplicaciuxf3n-al-caso-rca-y-profand}}

En la sucesión de la ex-Alpesca, el Estado Provincial impuso la
aplicación estricta de esta tutela. Red Chamber Argentina debió asumir
las deudas laborales preexistentes de los trabajadores de
Alpesca.\textsuperscript{4} Posteriormente, cuando el Gobierno de Chubut
rescindió el contrato con RCA, se aseguró que el nuevo operador
(Profand-Consermar) se comprometiera a

\textbf{contratar a la totalidad de los trabajadores, reconocerles la
antigüedad devengada y mantener sus condiciones
laborales}.\textsuperscript{1}

\hypertarget{el-hito-profand-la-tutela-laboral-como-condiciuxf3n-de-poluxedtica-puxfablica}{%
\paragraph{2. El Hito Profand: La Tutela Laboral como Condición de
Política
Pública}\label{el-hito-profand-la-tutela-laboral-como-condiciuxf3n-de-poluxedtica-puxfablica}}

Dado que los bienes de Alpesca siguen bajo la figura de "utilidad
pública sujetos a expropiación" \textsuperscript{2}, la provincia de
Chubut ejerció su potestad administrativa para garantizar la continuidad
laboral no solo como un derivado legal (Art. 225 LCT), sino como una

\textbf{Condición de Política Pública} impuesta al nuevo inversor. La
transferencia del establecimiento se ejecutó con la finalidad explícita
de dejar atrás años de desorden y garantizar un futuro más justo para
los chubutenses, forzando al capital inversor a internalizar los costos
sociales de la crisis anterior.\textsuperscript{1} Esto demuestra que,
en situaciones de emergencia económica y política, la ley laboral
argentina puede ser excepcionalmente justa para el trabajador, ya que el
riesgo empresarial (la mala gestión o el incumplimiento de RCA) se
transfiere obligatoriamente al sucesor solvente (Profand).

\hypertarget{b.-mecanismos-de-evasiuxf3n-y-fraude-a-la-ley}{%
\subsubsection{B. Mecanismos de Evasión y Fraude a la
Ley}\label{b.-mecanismos-de-evasiuxf3n-y-fraude-a-la-ley}}

Las denuncias de incumplimiento laboral a menudo se solapan con el uso
de figuras jurídicas para evadir responsabilidades, generando fraude.

\hypertarget{interposiciuxf3n-y-solidaridad-arts.-29-y-30-lct}{%
\paragraph{1. Interposición y Solidaridad (Arts. 29 y 30
LCT)}\label{interposiciuxf3n-y-solidaridad-arts.-29-y-30-lct}}

La LCT consagra la figura de la interposición de personas (Art. 29 LCT),
una protección que posibilita al trabajador reclamar al sujeto
interpuesto o al verdadero empleador.\textsuperscript{14} Esto es
relevante cuando se utiliza un "empresario aparente" que carece de
autonomía o medios propios, práctica común en la cesión ilegal de
trabajadores.\textsuperscript{15}

Además, el Artículo 30 establece la responsabilidad solidaria para los
subcontratistas cuando las tareas subcontratadas corresponden a la
actividad normal y específica propia del
establecimiento.\textsuperscript{16} En una planta pesquera, la
subcontratación de tareas de procesamiento esenciales (como el fileteado
o la elaboración) puede ser considerada parte de la actividad normal y
específica.\textsuperscript{16}

\hypertarget{consecuencia-legal-del-fraude}{%
\paragraph{2. Consecuencia Legal del
Fraude}\label{consecuencia-legal-del-fraude}}

La jurisprudencia evalúa criterios como la autonomía del objeto de la
contrata, la justificación técnica, y si las tareas de los trabajadores
contratados son idénticas a las de la plantilla
principal.\textsuperscript{15} Si se comprueba el fraude a la ley, la
consecuencia es el desplazamiento de la norma utilizada para evadir la
ley (la norma de cobertura), y se establece la vigencia plena de la
norma de orden público preterida.\textsuperscript{17} Esto implicaría
que, si RCA utilizaba subcontratistas para evadir el CCT 372/04
\textsuperscript{3}, estos trabajadores tendrían derecho a la aplicación
plena del convenio colectivo sectorial principal.\textsuperscript{17}

\hypertarget{iv.-tutela-sindical-pruxe1cticas-desleales-y-rol-de-la-intervenciuxf3n-estatal}{%
\subsection{IV. Tutela Sindical, Prácticas Desleales y Rol de la
Intervención
Estatal}\label{iv.-tutela-sindical-pruxe1cticas-desleales-y-rol-de-la-intervenciuxf3n-estatal}}

La conflictividad en la industria pesquera de Chubut ha sido
históricamente alta.\textsuperscript{18} El sistema legal argentino
provee herramientas sólidas para la tutela de la acción gremial.

\hypertarget{a.-la-libertad-sindical-como-derecho-fundamental-y-su-tutela-ley-23.551}{%
\subsubsection{A. La Libertad Sindical como Derecho Fundamental y su
Tutela (Ley
23.551)}\label{a.-la-libertad-sindical-como-derecho-fundamental-y-su-tutela-ley-23.551}}

La Ley 23.551, junto con la LCT, garantiza la libertad sindical y
protege a los representantes gremiales con estabilidad
laboral.\textsuperscript{19}

\hypertarget{funciones-de-tutela-y-prohibiciuxf3n-de-pruxe1cticas-desleales}{%
\paragraph{1. Funciones de Tutela y Prohibición de Prácticas
Desleales}\label{funciones-de-tutela-y-prohibiciuxf3n-de-pruxe1cticas-desleales}}

Los delegados del personal y las comisiones internas ejercen la
representación de los trabajadores ante el
empleador.\textsuperscript{21} La ley prohíbe taxativamente al empleador
adoptar represalias contra los trabajadores por su participación en
medidas legítimas de acción sindical o por haber intervenido en
procedimientos de juzgamiento de prácticas
desleales.\textsuperscript{22} La defensa de la libertad sindical y la
estabilidad han sido consagradas por la Corte Suprema argentina,
estableciendo un estándar de tutela elevado, desconocido antes de fallos
como "ATE" y "Álvarez".\textsuperscript{23}

\hypertarget{b.-el-papel-del-estado-de-chubut-como-actor-de-tutela-activa}{%
\subsubsection{B. El Papel del Estado de Chubut como Actor de Tutela
Activa}\label{b.-el-papel-del-estado-de-chubut-como-actor-de-tutela-activa}}

El caso RCA/Alpesca es notable por la decisión política de la autoridad
administrativa de intervenir activamente para hacer cumplir la ley
laboral.

\hypertarget{la-decisiuxf3n-de-extinciuxf3n-y-el-incumplimiento-comprobado}{%
\paragraph{1. La Decisión de Extinción y el Incumplimiento
Comprobado}\label{la-decisiuxf3n-de-extinciuxf3n-y-el-incumplimiento-comprobado}}

El Gobierno Provincial de Chubut tomó la decisión de declarar la
extinción del contrato con RCA tras una auditoría transparente que
comprobó incumplimientos, lo que implica la validación de las denuncias
laborales y contractuales.\textsuperscript{1} Esta acción es una
manifestación directa de la tutela estatal, que utiliza el poder
regulatorio de las concesiones para penalizar la violación de
obligaciones sociales.

\hypertarget{la-paradoja-de-la-justicia-eficaz-y-la-lenta-judicializaciuxf3n}{%
\paragraph{2. La Paradoja de la Justicia Eficaz y la Lenta
Judicialización}\label{la-paradoja-de-la-justicia-eficaz-y-la-lenta-judicializaciuxf3n}}

El conflicto revela una debilidad estructural del sistema: la lentitud
del proceso judicial. El proceso de expropiación de los bienes de
Alpesca, y el litigio civil y comercial asociado, está en curso desde
2014 sin que se haya dictado sentencia en primera
instancia.\textsuperscript{13} Esta dilatación judicial prolonga la
inseguridad jurídica para todas las partes.

Sin embargo, el desenlace del conflicto laboral de RCA no esperó la
resolución judicial. La certeza para los trabajadores fue restaurada por
la \textbf{decisión administrativa-política} del Poder Ejecutivo
Provincial (el decreto de extinción y el acuerdo con
Profand).\textsuperscript{1} Esta rápida y determinante intervención
demostró ser más eficaz para garantizar la continuidad laboral en una
crisis sistémica que el sistema judicial ordinario, el cual se había
estancado durante años. La ley, en este contexto, fue aplicada de manera
justa y efectiva gracias a la voluntad política de la autoridad
provincial de intervenir.

\hypertarget{v.-evaluaciuxf3n-cruxedtica-es-justa-la-legislaciuxf3n-laboral-argentina-en-la-pruxe1ctica}{%
\subsection{V. Evaluación Crítica: ¿Es Justa la Legislación Laboral
Argentina en la
Práctica?}\label{v.-evaluaciuxf3n-cruxedtica-es-justa-la-legislaciuxf3n-laboral-argentina-en-la-pruxe1ctica}}

La evaluación de la justicia de la legislación laboral argentina en el
sector pesquero debe confrontar la robustez de sus principios (justicia
formal) con su aplicación práctica y los desafíos estructurales
(justicia material).

\hypertarget{a.-balance-cruxedtico-sobre-la-justicia-y-eficacia-regulatoria-en-la-pesca-argentina}{%
\subsubsection{A. Balance Crítico sobre la Justicia y Eficacia
Regulatoria en la Pesca
Argentina}\label{a.-balance-cruxedtico-sobre-la-justicia-y-eficacia-regulatoria-en-la-pesca-argentina}}

La legislación laboral argentina es dual en su impacto. Sus principios
son de avanzada, pero su eficacia está limitada por la realidad
socioeconómica del sector.

\begin{longtable}[]{@{}
  >{\raggedright\arraybackslash}p{(\columnwidth - 6\tabcolsep) * \real{0.2500}}
  >{\raggedright\arraybackslash}p{(\columnwidth - 6\tabcolsep) * \real{0.2500}}
  >{\raggedright\arraybackslash}p{(\columnwidth - 6\tabcolsep) * \real{0.2500}}
  >{\raggedright\arraybackslash}p{(\columnwidth - 6\tabcolsep) * \real{0.2500}}@{}}
\toprule
\begin{minipage}[b]{\linewidth}\raggedright
\textbf{Principio Evaluado}
\end{minipage} & \begin{minipage}[b]{\linewidth}\raggedright
\textbf{Aspecto Justo / Eficacia (Fortalezas)}
\end{minipage} & \begin{minipage}[b]{\linewidth}\raggedright
\textbf{Aspecto Injusto / Ineficacia (Debilidades)}
\end{minipage} & \begin{minipage}[b]{\linewidth}\raggedright
\textbf{Referencia}
\end{minipage} \\
\begin{minipage}[b]{\linewidth}\raggedright
Principio Protectorio y Continuidad
\end{minipage} & \begin{minipage}[b]{\linewidth}\raggedright
La ley garantiza la estabilidad del empleo formal y la asunción de
obligaciones por sucesores, como lo demostró la transferencia
obligatoria a Profand.\textsuperscript{1}
\end{minipage} & \begin{minipage}[b]{\linewidth}\raggedright
La protección solo abarca al sector formal. El 62-80\% de informalidad
regional en la pesca impide la aplicación de estos derechos a la mayoría
de la fuerza laboral.\textsuperscript{24}
\end{minipage} & \begin{minipage}[b]{\linewidth}\raggedright
\textsuperscript{1}
\end{minipage} \\
\begin{minipage}[b]{\linewidth}\raggedright
Rol del Estado/Regulación
\end{minipage} & \begin{minipage}[b]{\linewidth}\raggedright
El Estado Provincial intervino activamente para extinguir el contrato de
RCA y forzar la continuidad laboral bajo condiciones preexistentes,
priorizando la fuente de trabajo.\textsuperscript{1}
\end{minipage} & \begin{minipage}[b]{\linewidth}\raggedright
La aplicación de la ley efectiva está sujeta a la voluntad política y la
dilación del proceso judicial de expropiación (pendiente desde 2014),
generando inseguridad jurídica de largo plazo.\textsuperscript{1}
\end{minipage} & \begin{minipage}[b]{\linewidth}\raggedright
\textsuperscript{1}
\end{minipage} \\
\begin{minipage}[b]{\linewidth}\raggedright
Negociación Colectiva
\end{minipage} & \begin{minipage}[b]{\linewidth}\raggedright
Existe un robusto marco para la negociación de CCTs que establece
condiciones sectoriales específicas.\textsuperscript{3}
\end{minipage} & \begin{minipage}[b]{\linewidth}\raggedright
Los CCTs son criticados por su antigüedad y rigidez, lo que puede
motivar la búsqueda activa de fraude y la limitación de la
competitividad.\textsuperscript{11}
\end{minipage} & \begin{minipage}[b]{\linewidth}\raggedright
\textsuperscript{3}
\end{minipage} \\
\midrule
\endhead
\bottomrule
\end{longtable}

\hypertarget{b.-la-tensiuxf3n-entre-derecho-y-realidad-socio-productiva}{%
\subsubsection{B. La Tensión entre Derecho y Realidad
Socio-Productiva}\label{b.-la-tensiuxf3n-entre-derecho-y-realidad-socio-productiva}}

\hypertarget{la-paradoja-de-la-informalidad}{%
\paragraph{1. La Paradoja de la
Informalidad}\label{la-paradoja-de-la-informalidad}}

La crítica más profunda al sistema legal argentino no es su contenido,
sino su alcance. La LCT es un cuerpo normativo diseñado para el
trabajador formal. Sin embargo, en el sector pesquero de América Latina,
el grado de informalidad se estima en un rango de 62 a 80 por
ciento.\textsuperscript{24} Estos trabajadores carecen de contrato
formal, seguridad social, acceso a mecanismos de protección de salud y
seguridad en el trabajo (SST).\textsuperscript{24} Por lo tanto, aunque
la legislación es doctrinalmente

\textbf{justa} y robusta para la minoría formal (los 17.000 trabajadores
comprendidos por el CCT de Procesamiento \textsuperscript{3}), es
funcionalmente

\textbf{ineficaz e injusta} para la gran mayoría del sector que queda al
margen de la tutela y no logra la "hominización" de la sociedad que
buscan los sindicatos modernos.\textsuperscript{19}

\hypertarget{el-conflicto-de-la-seguridad-juruxeddica}{%
\paragraph{2. El Conflicto de la Seguridad
Jurídica}\label{el-conflicto-de-la-seguridad-juruxeddica}}

La empresa RCA lamentó que el discurso político y mediático se desviara
hacia la desinformación, reafirmando su voluntad de defender sus
derechos.\textsuperscript{13} El sector empresario en general critica la
inseguridad jurídica que emana de regulaciones amplias o las
intervenciones estatales.\textsuperscript{25} No obstante, el análisis
del caso demuestra una dualidad: la misma intervención estatal que RCA
percibió como "animosidad" fue la que restauró la estabilidad laboral y
la seguridad económica para los trabajadores de la
ex-Alpesca.\textsuperscript{1} La seguridad jurídica, vista desde la
óptica protectoria, implica que el Estado no puede permitir que la
inestabilidad empresarial ponga en riesgo la fuente de trabajo, incluso
si esto implica una intervención forzosa en la gestión de los activos
concesionados.

\hypertarget{vi.-conclusiones-y-recomendaciones-estratuxe9gicas}{%
\subsection{VI. Conclusiones y Recomendaciones
Estratégicas}\label{vi.-conclusiones-y-recomendaciones-estratuxe9gicas}}

\hypertarget{a.-suxedntesis-del-diagnuxf3stico-sobre-el-caso-rca-el-triunfo-de-la-tutela-activa}{%
\subsubsection{A. Síntesis del Diagnóstico sobre el Caso RCA: El Triunfo
de la Tutela
Activa}\label{a.-suxedntesis-del-diagnuxf3stico-sobre-el-caso-rca-el-triunfo-de-la-tutela-activa}}

El caso de Red Chamber Argentina, en el marco de la sucesión de la
ex-Alpesca, sirve como un ejemplo de cómo la articulación de la LCT
(Arts. 75, 225) y el derecho administrativo (Ley I N° 527) puede lograr
una tutela laboral excepcionalmente fuerte. Las denuncias de
incumplimiento, validadas por la auditoría provincial, llevaron a la
extinción del contrato y a la imposición de continuidad laboral al
sucesor (Profand).\textsuperscript{1} Esto evitó que los costos de la
mala gestión y el riesgo empresarial recayeran sobre los trabajadores,
logrando que el principio protectorio se impusiera efectivamente sobre
los intereses corporativos.

\hypertarget{b.-conclusiuxf3n-sobre-la-justicia-del-marco-regulatorio-laboral-argentino}{%
\subsubsection{B. Conclusión sobre la Justicia del Marco Regulatorio
Laboral
Argentino}\label{b.-conclusiuxf3n-sobre-la-justicia-del-marco-regulatorio-laboral-argentino}}

La legislación laboral argentina, en su formulación y principios, es
\textbf{justa} y protectoria. El sistema legal provee las herramientas
adecuadas para la defensa de los derechos, la dignidad y la continuidad
del empleo formal. Sin embargo, el sistema es \textbf{parcialmente
injusto} e \textbf{ineficaz} en términos estructurales. La justicia
efectiva en el sector pesquero depende de la voluntad política y la
acción administrativa decisiva (como la demostrada por el Gobierno de
Chubut) para intervenir en crisis sectoriales complejas, y no alcanza a
resolver la profunda crisis de informalidad que afecta a la gran mayoría
de los trabajadores del sector.\textsuperscript{24} La ley es robusta
para los formalizados, pero es una promesa incumplida para los
informales.

\hypertarget{c.-recomendaciones-de-poluxedtica-puxfablica}{%
\subsubsection{C. Recomendaciones de Política
Pública}\label{c.-recomendaciones-de-poluxedtica-puxfablica}}

\begin{enumerate}
\def\labelenumi{\arabic{enumi}.}
\item
  \begin{quote}
  \textbf{Refuerzo de la Fiscalización en el Sector Informal:} Se
  recomienda redirigir los esfuerzos de fiscalización hacia las
  prácticas de interposición y fraude (Arts. 29 y 30 LCT) en las tareas
  esenciales de procesamiento, con el objetivo de reducir la
  informalidad sectorial que alcanza hasta el 80\%.\textsuperscript{24}
  El cumplimiento del Art. 75 (Seguridad e Higiene) debe ser una
  prioridad constante en la fiscalización de plantas.
  \end{quote}
\item
  \begin{quote}
  \textbf{Agilización de la Tutela Administrativa:} Se debe formalizar
  la capacidad de la autoridad administrativa provincial (Trabajo y
  Pesca) para intervenir y resolver conflictos laborales masivos
  asociados a concesiones de recursos, asegurando que las decisiones
  políticas de continuidad laboral sean inmediatas y no queden
  supeditadas a la dilación de procesos judiciales de largo aliento
  (como el caso Alpesca/Expropiación).\textsuperscript{13}
  \end{quote}
\item
  \begin{quote}
  \textbf{Modernización Negociada de los CCTs:} Es imperativo fomentar
  el diálogo social entre sindicatos y cámaras empresarias para revisar
  los CCTs antiguos.\textsuperscript{11} El objetivo debe ser lograr un
  equilibrio que mantenga la protección esencial de los trabajadores
  (orden público laboral) mientras se incorpora la flexibilidad
  necesaria para mejorar la competitividad, eliminando el incentivo para
  que las empresas busquen activamente el fraude como mecanismo de
  reducción de costos.
  \end{quote}
\end{enumerate}

\hypertarget{obras-citadas}{%
\paragraph{Obras citadas}\label{obras-citadas}}

\begin{enumerate}
\def\labelenumi{\arabic{enumi}.}
\item
  \begin{quote}
  Tras el acuerdo con Provincia, Profand invertirá ... - Noticias
  Chubut, fecha de acceso: octubre 5, 2025,
  \href{https://noticias.chubut.gob.ar/notas/gobierno/3594/tras-el-acuerdo-con-provincia-profand-invertira-mas-de-70-millones-de-euros-en-la-ex-alpesca-y-garantizara-continuidad-laboral-y-nuevos-empleos.html}{\uline{https://noticias.chubut.gob.ar/notas/gobierno/3594/tras-el-acuerdo-con-provincia-profand-invertira-mas-de-70-millones-de-euros-en-la-ex-alpesca-y-garantizara-continuidad-laboral-y-nuevos-empleos.html}}
  \end{quote}
\item
  \begin{quote}
  Declárase de Utilidad Pública y Sujeto a Expropiación de los Bienes
  Inmuebles, Buques, Permisos y Cuotas de la Empresa Alpesca SA y Ap
  Holding SA - Visualización de leyes, fecha de acceso: octubre 5, 2025,
  \href{https://digesto.legislaturadelchubut.gob.ar/public/rama/1/ley/527}{\uline{https://digesto.legislaturadelchubut.gob.ar/public/rama/1/ley/527}}
  \end{quote}
\item
  \begin{quote}
  CONVENIO COLECTIVO - Gobierno de Santa Fe, fecha de acceso: octubre 5,
  2025,
  \href{https://www.santafe.gov.ar/index.php/web/content/download/67083/325129/file/CCT+372-2004+Alimentaci\%F3n+(Pesca,+Procesamiento+y+Elaboraci\%F3n+de+Derivados).pdf}{\uline{https://www.santafe.gov.ar/index.php/web/content/download/67083/325129/file/CCT+372-2004+Alimentaci\%F3n+(Pesca,+Procesamiento+y+Elaboraci\%F3n+de+Derivados).pdf}}
  \end{quote}
\item
  \begin{quote}
  Red Chamber Argentina deberá asumir las deudas laborales de los
  trabajadores de Alpesca - Pescare, fecha de acceso: octubre 5, 2025,
  \href{https://pescare.com.ar/red-chamber-argentina-debera-asumir-las-deudas-laborales-de-los-trabajadores-de-alpesca/}{\uline{https://pescare.com.ar/red-chamber-argentina-debera-asumir-las-deudas-laborales-de-los-trabajadores-de-alpesca/}}
  \end{quote}
\item
  \begin{quote}
  Ley 20744 - InfoLeg - Información Legislativa, fecha de acceso:
  octubre 5, 2025,
  \href{https://servicios.infoleg.gob.ar/infolegInternet/verNorma.do?id=25552}{\uline{https://servicios.infoleg.gob.ar/infolegInternet/verNorma.do?id=25552}}
  \end{quote}
\item
  \begin{quote}
  Ley de Contratos de Trabajo nº 20.744 Conocer los principios
  protectorios del derecho laboral. Identificar los derechos y las -
  UNLP, fecha de acceso: octubre 5, 2025,
  \href{https://unlp.edu.ar/wp-content/uploads/0/27600/8191bd05c52411d18dbd5a335b8e0adf.pdf}{\uline{https://unlp.edu.ar/wp-content/uploads/0/27600/8191bd05c52411d18dbd5a335b8e0adf.pdf}}
  \end{quote}
\item
  \begin{quote}
  Ley de Contrato de Trabajo - Texto actualizado \textbar{}
  Argentina.gob.ar, fecha de acceso: octubre 5, 2025,
  \href{https://www.argentina.gob.ar/normativa/nacional/ley-20744-25552/actualizacion}{\uline{https://www.argentina.gob.ar/normativa/nacional/ley-20744-25552/actualizacion}}
  \end{quote}
\item
  \begin{quote}
  Régimen de Contrato de Trabajo - Jus.gob.ar - Infoleg, fecha de
  acceso: octubre 5, 2025,
  \href{https://servicios.infoleg.gob.ar/infolegInternet/anexos/25000-29999/25552/texact.htm}{\uline{https://servicios.infoleg.gob.ar/infolegInternet/anexos/25000-29999/25552/texact.htm}}
  \end{quote}
\item
  \begin{quote}
  fecha de acceso: octubre 5, 2025,
  \href{https://www.saij.gob.ar/obligaciones-empleador-condiciones-trabajo-seguridad-trabajador-salud-trabajador-sue0025783/123456789-0abc-defg3875-200esoiramus?\&o=52\&f=Total\%7CFecha\%7CEstado+de+Vigencia\%5B5,1\%5D\%7CTema\%5B5,1\%5D\%7COrganismo\%5B5,1\%5D\%7CAutor\%5B5,1\%5D\%7CJurisdicci\%EF\%BF\%BDn/Nacional\%7CTribunal\%5B5,1\%5D\%7CPublicaci\%EF\%BF\%BDn\%5B5,1\%5D\%7CColecci\%EF\%BF\%BDn+tem\%EF\%BF\%BDtica\%5B5,1\%5D\%7CTipo+de+Documento/Jurisprudencia\&t=176066\#:~:text=El\%20art\%2075\%20de\%20la,de\%20trabajo\%20dignas\%20y\%20seguras.}{\uline{https://www.saij.gob.ar/obligaciones-empleador-condiciones-trabajo-seguridad-trabajador-salud-trabajador-sue0025783/123456789-0abc-defg3875-200esoiramus?\&o=52\&f=Total\%7CFecha\%7CEstado\%20de\%20Vigencia\%5B5\%2C1\%5D\%7CTema\%5B5\%2C1\%5D\%7COrganismo\%5B5\%2C1\%5D\%7CAutor\%5B5\%2C1\%5D\%7CJurisdicci\%F3n/Nacional\%7CTribunal\%5B5\%2C1\%5D\%7CPublicaci\%F3n\%5B5\%2C1\%5D\%7CColecci\%F3n\%20tem\%E1tica\%5B5\%2C1\%5D\%7CTipo\%20de\%20Documento/Jurisprudencia\&t=176066\#:\textasciitilde:text=El\%20art\%2075\%20de\%20la,de\%20trabajo\%20dignas\%20y\%20seguras.}}
  \end{quote}
\item
  \begin{quote}
  Convenios Colectivos de Trabajo en el Sistema Pesquero Argentino: un
  análisis de los CCT del procesamiento de materia prima - Revista
  Paginas, fecha de acceso: octubre 5, 2025,
  \href{https://revistapaginas.unr.edu.ar/index.php/RevPaginas/article/view/802/1035}{\uline{https://revistapaginas.unr.edu.ar/index.php/RevPaginas/article/view/802/1035}}
  \end{quote}
\item
  \begin{quote}
  Entre los desafíos de la pesca también está el de combatir la
  desinformación, fecha de acceso: octubre 5, 2025,
  \href{https://revistapuerto.com.ar/2025/09/entre-los-desafios-de-la-pesca-tambien-esta-el-de-combatir-la-desinformacion/}{\uline{https://revistapuerto.com.ar/2025/09/entre-los-desafios-de-la-pesca-tambien-esta-el-de-combatir-la-desinformacion/}}
  \end{quote}
\item
  \begin{quote}
  Ley 24.922 Régimen Federal De Pesca - Texto actualizado \textbar{}
  Argentina.gob.ar, fecha de acceso: octubre 5, 2025,
  \href{https://www.argentina.gob.ar/normativa/nacional/48357/actualizacion}{\uline{https://www.argentina.gob.ar/normativa/nacional/48357/actualizacion}}
  \end{quote}
\item
  \begin{quote}
  Se aguardan definiciones en torno a la continuidad de Red Chamber -
  Revista Puerto, fecha de acceso: octubre 5, 2025,
  \href{https://revistapuerto.com.ar/2025/08/se-aguardan-definiciones-en-torno-a-la-continuidad-de-red-chamber/}{\uline{https://revistapuerto.com.ar/2025/08/se-aguardan-definiciones-en-torno-a-la-continuidad-de-red-chamber/}}
  \end{quote}
\item
  \begin{quote}
  fecha de acceso: octubre 5, 2025,
  \href{https://aldiaargentina.microjuris.com/2024/08/02/doctrina-la-responsabilidad-solidaria-en-la-reforma-laboral-art-90-de-la-ley-27-742/\#:~:text=El\%20art.,las\%20indemnizaciones\%20que\%20le\%20corresponden.}{\uline{https://aldiaargentina.microjuris.com/2024/08/02/doctrina-la-responsabilidad-solidaria-en-la-reforma-laboral-art-90-de-la-ley-27-742/\#:\textasciitilde:text=El\%20art.,las\%20indemnizaciones\%20que\%20le\%20corresponden.}}
  \end{quote}
\item
  \begin{quote}
  Cesión ilegal de trabajadores: diferenciación con la subcontratación
  de obra y servicios, fecha de acceso: octubre 5, 2025,
  \href{https://www.jover-abogados.com/cesion-ilegal-trabajadores-diferenciacion-la/}{\uline{https://www.jover-abogados.com/cesion-ilegal-trabajadores-diferenciacion-la/}}
  \end{quote}
\item
  \begin{quote}
  Dossier: Subcontratación Laboral - SAIJ, fecha de acceso: octubre 5,
  2025,
  \href{https://www.saij.gob.ar/docs-f/dossier-f/subcontratacion_laboral.pdf}{\uline{https://www.saij.gob.ar/docs-f/dossier-f/subcontratacion\_laboral.pdf}}
  \end{quote}
\item
  \begin{quote}
  fraude y subcontratación. - Ministerio de Trabajo de la Provincia de
  Buenos Aires, fecha de acceso: octubre 5, 2025,
  \href{https://www.trabajo.gba.gov.ar/paneles-y-ponencias/fraude-laboral/barreranicholson.pdf}{\uline{https://www.trabajo.gba.gov.ar/paneles-y-ponencias/fraude-laboral/barreranicholson.pdf}}
  \end{quote}
\item
  \begin{quote}
  Conflictos en torno a la actividad pesquera en la Patagonia argentina:
  de 1997 a 2007 en el noreste de Chubut - Redalyc, fecha de acceso:
  octubre 5, 2025,
  \href{https://www.redalyc.org/journal/3873/387370671003/html/}{\uline{https://www.redalyc.org/journal/3873/387370671003/html/}}
  \end{quote}
\item
  \begin{quote}
  Dubra.pdf - UNTREF, fecha de acceso: octubre 5, 2025,
  \href{http://untref.edu.ar/documentos/tesisposgrados/Dubra.pdf}{\uline{http://untref.edu.ar/documentos/tesisposgrados/Dubra.pdf}}
  \end{quote}
\item
  \begin{quote}
  Aportes para el estudio de la tutela sindical en el régimen jurídico
  vigente - SEDICI, fecha de acceso: octubre 5, 2025,
  \href{http://sedici.unlp.edu.ar/bitstream/handle/10915/58172/Documento_completo__.pdf-PDFA.pdf?sequence=1\&isAllowed=y}{\uline{http://sedici.unlp.edu.ar/bitstream/handle/10915/58172/Documento\_completo\_\_.pdf-PDFA.pdf?sequence=1\&isAllowed=y}}
  \end{quote}
\item
  \begin{quote}
  Ley 23 - Jus.gob.ar - Infoleg, fecha de acceso: octubre 5, 2025,
  \href{https://servicios.infoleg.gob.ar/infolegInternet/anexos/20000-24999/20993/texact.htm}{\uline{https://servicios.infoleg.gob.ar/infolegInternet/anexos/20000-24999/20993/texact.htm}}
  \end{quote}
\item
  \begin{quote}
  Ley de Asociaciones Sindicales - Ley Nº 23.551, fecha de acceso:
  octubre 5, 2025,
  \href{https://www.psicologosgcaba.org.ar/wp-content/uploads/ley-23551.pdf}{\uline{https://www.psicologosgcaba.org.ar/wp-content/uploads/ley-23551.pdf}}
  \end{quote}
\item
  \begin{quote}
  ESTRUCTURA NORMATIVA Y GARANTIAS PARA EL EJERCICIO DE UN DERECHO
  FUNDAMENTAL: LA LIBERTAD SINDICAL EN ARGENTINA, 2003-2011, fecha de
  acceso: octubre 5, 2025,
  \href{https://www.derecho.uba.ar/institucional/2014-ubacyt-version-final.pdf}{\uline{https://www.derecho.uba.ar/institucional/2014-ubacyt-version-final.pdf}}
  \end{quote}
\item
  \begin{quote}
  Estudio de la OIT revela desafíos y avances hacia el trabajo decente
  en los sectores pesca y acuicultura en América Latina y el Caribe
  \textbar{} International Labour Organization, fecha de acceso: octubre
  5, 2025,
  \href{https://www.ilo.org/es/resource/news/estudio-de-la-oit-revela-desafios-y-avances-hacia-el-trabajo-decente-en-los-0}{\uline{https://www.ilo.org/es/resource/news/estudio-de-la-oit-revela-desafios-y-avances-hacia-el-trabajo-decente-en-los-0}}
  \end{quote}
\item
  \begin{quote}
  Desafíos que la pesca va a superar - Pescare, fecha de acceso: octubre
  5, 2025,
  \href{https://pescare.com.ar/desafios-que-la-pesca-va-a-superar/}{\uline{https://pescare.com.ar/desafios-que-la-pesca-va-a-superar/}}
  \end{quote}
\end{enumerate}

\end{document}
